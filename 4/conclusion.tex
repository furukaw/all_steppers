\section{この章のまとめ}
\label{section:conclusion}

ステッパを実装するためには、
コンテキストの情報を保持しながら部分式を再帰的に実行するインタプリタを作ればよい。
\ref{chapter:try-with} 章で行った方法では言語ごとにコンテキストを表すデータ型を
考えた上でインタプリタに実行の流れに従った新しい引数を付け足す作業が必要だったが、
本研究では通常のインタプリタをCPS変換および非関数化するという機械的な操作で
コンテキストの型およびコンテキストの情報を保持するインタプリタ関数を導出した。

その方法で、継続を明示的に扱える algebraic effects を含む言語に対するステッパを実装し、他の例外処理機能である try-with や shift/reset を含む言語についても同様の変換ができることを確認した。
