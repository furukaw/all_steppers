\section{授業の内容}
\label{section:experiment__course}
% The ``Functional Programming'' course teaches
% how to program with functions and types, covering
% basic topics such as recursion, datatypes, effects, and modules.
「関数型言語」は OCaml を用いて関数型プログラミングを学ぶ授業であり、
再帰、データ型、副作用、モジュールなどの基本的な言語機能を扱う。
% The course consists of 15, weekly lab sessions, and each session
%  consists of 90 minutes lab-style class per week.
学期全体を通して、学生はダイクストラ法に基づいて最短路問題を解くプログラムを作成する。
授業は週に1回、90分間で、全15回で構成されている。
% (Many students remain in the lab after 90 minutes up until around 150
% minutes.)
(授業終了後も教室に残って作業を続ける学生も多い。)
% Throughout the course, students build a program that searches for
% the shortest path based on Dijkstra's algorithm.
% The participants of the course are second-year undergraduate students
% majoring in computer science (around 40 students each year).
% All students enter this course after a CS 1 course in the C
% programming language.
受講者は情報科学を専攻する学部2年生で、毎年40人程度である。
全ての学生がこの授業より前にC言語によるプログラミングを経験している。

% The course is taught in a ``flipped classroom'' style.
この授業では「反転授業」を行なっている。
反転授業とは、学生が各自内容を予習し、授業時間は講義を行わず演習などに活用する授業形態である。
% Before every meeting, students are asked to study assigned readings
% and videos prepared by the instructor and answer simple quizzes.
この授業では、学生は毎週の授業の前に、この授業に向けた教科書 \ref{Asai07} と教員が作成した動画による予習をし、
インターネット上で数問の簡単な問題に解答する。
% In the classroom, they practice the newly covered topics
% through exercises, with assistance of the instructor as well as five
% to six teaching assistants (including the first and second authors).  
授業時間内には、学生はその回に新しく取り上げた内容に関する演習問題に取り組み、
教員および5〜8人のティーチングアシスタントに質問することができる。

% The exercises include simple practice problems and report problems.
演習問題はほとんどが要求を満たす関数を実装するというものであり、
「練習問題」と「レポート課題」に分かれている。
% The former are for confirming students' understanding of the
% topics and are expected to be completed within a class.
練習問題は学生がどの程度その回の内容を理解しているか確認するための簡単な問題で、
授業時間内に解くことを期待している。
% The latter problems (for credit) are due in one week.
レポート課題は評定に利用する問題■■で、締め切りが1週間後に設定されている。

授業をする演習室では常に OCaml プログラム実行のログをとっている。
% Whenever a student executes a program, by either step execution or
% standard execution, the program as well as its execution log (syntax
% errors, type errors, or the result of execution) are recorded.
学生がプログラムを実行する時には必ず、ステッパによる実行と通常の実行のどちらも、
プログラム全文と実行ログが記録される。

% For most of the problems (up to the 12th week), we provide a check system
% where students can submit their solutions to see whether they pass the
% given tests.
解答プログラムが合っているかどうかを確認する方法の 1 つとして、
「チェックシステム」を用意してある。
チェックシステムは教員が用意した単体テストに通るかどうかを確認できるプログラムで、
問題ごとに用意されており、12 週目までのほとんどの問題を対象にしている。
各学生が各問題のチェックシステムに初めて通った時には学生の ID が記録されるようになっており、
% To earn points for report problems, students are required to have their
% programs pass the check system.
締め切りまでに 1 度以上チェックシステムに通ることが、レポート課題の各問で点を得る条件に含まれている。
