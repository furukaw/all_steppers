\section{実際に使用したステッパ}
\label{section:experiment__stepper}

我々は、OCaml の一部の構文を対象としたステッパを実装した。
% In Figure \ref{figure:ocamlstep},
% we show a reduction sequence produced by the actual stepping evaluator.
OCaml ステッパが出力するステップ列を図 \ref{figure:ocamlstep} に示す。
% The evaluator supports the following syntactic constructs:
このステッパは以下の構文に対応している。

\begin{itemize}
% \item integers, floating point numbers, booleans, characters, strings
% \item lists, tuples, records
% \item user-defined datatypes
% \item conditionals, let-expressions, recursive functions, pattern-matching
% \item exception handling operators
% \item printing functions and sequential execution
% \item the List module, user-defined modules
% \item references, arrays
\item 整数、実数、真偽値、文字、文字列型
\item リスト、組、レコード
\item ユーザ定義型
\item 条件分岐、変数定義、再帰関数定義、パターンマッチ
\item 例外処理
\item 標準出力関数、逐次実行
\item List モジュール、ユーザ定義モジュール
\item 書き換え可能な変数、配列
\end{itemize}

% ref、配列を使うやつにする?

\begin{figure}
  \includegraphics[width=14cm]{6/longexample.eps}
%   \caption{Evaluating programs using the actual stepper}
  \caption{実際のステッパでプログラムを実行する様子}
  \label{figure:ocamlstep}
\end{figure}

\begin{figure}
  \includegraphics[width=10cm]{6/beforeskip.eps}
  \includegraphics[width=4.3cm]{6/afterskip.eps}
%   \caption{Skipping evaluation of the factorial function}
  \caption{階乗関数をスキップする様子}
  \label{figure:factskip}
\end{figure}

標準出力や書き換え可能な変数を含むプログラムでは、
標準出力された文字列や変数に格納された値をインタプリタが保持する必要があり、
そういった状態の情報はステッパプログラムの
書き換え可能なグローバル変数の中に格納することで実装した。

% To allow the user to adjust granularity of steps,
% we provide an option for skipping
% the evaluation of the current function application.
また、大きなプログラムのデバッグのためにステップ実行をするときに、
膨大な計算ステップを1つ1つ追って見るのは効率が悪く、
次第に実用に耐えるものではなくなってしまう。
そこで本研究では、「関数適用」を基準としてステップを飛ばす機能を追加した。

% Let us look at Figure \ref{figure:factskip},
% which shows skipping of the factorial function.
図 \ref{figure:factskip} に、階乗関数のスキップの様子を示す。
% By pressing the ``skip'' button,
% we can directly go from the program on the left to the one on the right,
% without seeing the intermediate steps that appear
% during the evaluation of the function's body.
左の状態で「skip」ボタンを押すと、関数の中身の実行ステップを見ずに右の画面に移ることができる。
% This feature helps us focus on the steps we are interested in,
% allowing us to grasp the overall flow of the execution.
これによって見たいステップだけに注目することができ、実行の全体の流れを把握しやすくなる。

% The skipping feature requires some modifications to
% the \texttt{eval} function (Figure \ref{figure:skipapp}).
このスキップ機能は、
\ref{chapter:try-with} 章で try-with に対応したステッパ
(図 \ref{figure:stepper}) の eval 関数を
図 \ref{figure:skipapp} のように変更することで実装できる。
% The idea is to sandwich the steps within an application between two strings:
本研究では、飛ばすステップ列を
% \texttt{(* Application n start *)} and \texttt{(* Application n end *)}.
\texttt{(* Application n start *)} と \texttt{(* Application n end *)}
という 2 つの文字列で挟む方法をとった。
% Here, \texttt{n} tells us at which step we have entered the application.
\texttt{n} は関数適用の実行が始まるステップの番号である。
% These strings are printed using the \texttt{apply\_start} and
% \texttt{apply\_end} functions,
% and help the Emacs Lisp program to hide unnecessary steps.
関数 \texttt{apply\_start} が前者を出力する関数、
関数 \texttt{apply\_end} が後者を出力する関数である。
表示処理をする Emacs Lisp プログラムがこれらの文字列を検索し、
間のステップを隠す処理をする。
% We show an example output sequence in Figure \ref{figure:skipping}.
ステッパ関数の出力は例えば図 \ref{figure:skipping} のようになる。

\begin{figure}
\begin{alltt}
let rec eval expr ctxt = match expr with
    ...
  | App (e1, e2) ->
    begin
      let v2 = eval e2 (add ctxt (CAppR e1)) in
      let v1 = eval e1 (add ctxt (CAppL v2)) in
      match v1 with
      | Lam (x, e) ->
        let e' = subst e x v2 in
        \colorbox{lightgray}{let apply_num = apply_start () in}                (* output start mark *)
        memo (App (v1, v2)) e' ctxt;
        let v = eval e' ctxt in
        \colorbox{lightgray}{apply_end apply_num;}                               (* output end mark *)
        v
      | _ -> failwith "not a function"
    end
  | ...
\end{alltt}
\caption{関数適用をスキップするためのステッパ関数}
\label{figure:skipapp}
\end{figure}

\begin{figure}
\texttt{(* Step 0 *) (f 4) + \colorbox{lightgreen}{10 * 100}\\
(* Step 1 *) (f 4) + \colorbox{purple}{1000}\\
(* Application 1 start *)\\
(* Step 1 *) \colorbox{lightgreen}{f 4} + 1000\\
(* Step 2 *) \colorbox{purple}{(4 * 2) - 1} + 1000\\
(* Step 2 *) \colorbox{lightgreen}{(4 * 2} - 1) + 1000\\
(* Step 3 *) \colorbox{purple}{(8} - 1) + 1000\\
(* Step 3 *) \colorbox{lightgreen}{8 - 1} + 1000\\
(* Step 4 *) \colorbox{purple}{7} + 1000\\
(* Application 1 end *)\\
(* Step 4 *) \colorbox{lightgreen}{7 + 1000}\\
(* Step 5 *) \colorbox{purple}{1007}}
\caption{関数適用をスキップするためのステッパ出力}
\label{figure:skipping}
\end{figure}
