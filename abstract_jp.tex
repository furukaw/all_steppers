\chapter*{要旨}

ステッパとはプログラミング教育やデバッグのために使うツールであり、
プログラムが代数的に書き換わる様子を出力することで実行過程を見せるものである。

これまでに Racket の教育用に制限した構文
などを対象にステッパが作られてきたが、
継続を扱うことができる代数的効果のような、
複雑なプログラム制御をする言語機能に対応したステッパは作られていなかった。
そのような複雑な機能を含むプログラムの挙動を理解するのは特に困難なので、
ステッパでプログラムの動きを観察できるようにしたい。
本研究では例外処理や継続操作の機能を含む言語に対応したステッパを実装した。

ステッパは簡約のたびにその時点でのプログラム全体を出力するインタプリタなので、
実行している部分式のコンテキスト(周りの式)の情報が常に必要になる。
このコンテキストの情報をどのように得るかというのが、
ステッパの実装における最大の問題である。
本研究ではコンテキストを自分で設計する方法と
機械的に導出する方法の 2 つを示す。

実装したステッパを利用してみると、
長いプログラムを入力すると実行に長い時間がかかってしまい
使用できないことが分かった。
その解決の為、
ステップ実行処理の終了を待たずに利用できる「incremental なステッパ」を実装した。

また、OCaml の一部の構文に対するステッパを大学の授業で学生に使用してもらい、
その実行ログなどからステッパの教育上の効果について考察し、
ステッパが有用な場面があることを示した。

\vspace{10mm}

{\bf キーワード:}
プログラミング教育、デバッグ、OCaml、代数的効果、CPS 変換、非関数化、インタプリタ\ 
