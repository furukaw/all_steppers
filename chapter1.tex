\chapter{序論}

この章では,どのような研究背景があり,そのためにどのような研究の動機が生
じ,自分の研究の目的が何であるかを明確にする.\\

科学技術論文の場合は,句読点は「、」「。」ではなく,「,」「.」を使う.
\\

卒論はしっかり書きましょう.就職する人は学生生活最後の一番大きな
レポートとなります.いつも言っていることですが,社会に出たら「成績」を評
価されるのではなく「業績」を評価されます.学生時代の学業の評価をされると
きに一番最初に聞かれるのは「卒論では何をやったの?」という質問です.卒業
論文をしっかりと書いて後で人に見せられる状態にしておくのは後々必ず役に立
ちます.\\

\vspace{2cm}

研究の動機と目的を記した後に,各章の章立てを説明する.

\vspace{1cm}
以下に本論文の構成を示す.{\bf 第2章}では,関連研究について述べる.
{\bf 第3章}では,第4章で自分が行った研究を遂行するために必要となった理論
などを説明する.
{\bf 第4章}では,
自分が提案する手法の説明をし,{\bf 第5章}で,シミュレーションや実験など
による具体的な結果を示すと共に研究成果に対する考察を述べる.最後に,{\bf
第6章}で,結論を述べる.
