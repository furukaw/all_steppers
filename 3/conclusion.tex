\section{この章のまとめ}
\label{section:try-with__conclusion}

この章では、例外処理の構文 try-with に対応したステッパ関数を、
インタプリタを拡張することによって実装する方法を示した。

まず、 try-with 構文に対する通常の big-step インタプリタは
実際の OCaml の try-with を用いて実装することができた。

ステッパ関数への拡張においては、各ステップでのプログラム全体を表示するために、
実行中の部分式のコンテキストの情報が必要である。
例外発生によって一度に捨てられる範囲のコンテキスト、
すなわち try 節の内側のコンテキストフレームをひとまとまりにして、
そのリストを要素に持つリストのような構造としてコンテキストを定義した。
それをインタプリタの再帰の構造にしたがって要素を足しながら引数に渡すようにすると
実行中の部分式のコンテキストの情報が得られるようになり、
プログラム全体が出力できるステッパ関数が得られた。
