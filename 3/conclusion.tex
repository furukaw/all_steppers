\section{おわりに}
\label{section:try-with__conclusion}

この章では、例外処理の構文 try-with に対応したステッパを、
インタプリタを拡張することによって実装する方法を示した。

まず、 try-with 構文に対する通常の big-step インタプリタは
実際の OCaml の try-with を用いて実装することができた。

ステッパへの拡張においては、各ステップでのプログラム全体を表示するために、
実行中の部分式のコンテキストの情報が必要である。
例外発生によって一度に捨てられる範囲のコンテキスト、
すなわち try 節の内側のコンテキストフレームのリストをひとまとめにして、
そのリストのような構造としてコンテキストを定義した。
それをインタプリタの再帰の構造にしたがって要素を足しながら引数に渡すようにすると
実行中の部分式のコンテキストの情報が得られるようになり、
プログラム全体が出力できるステッパが得られた。
