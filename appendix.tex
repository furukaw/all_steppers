\chapter{実験でもちいたデータ}

ここには,実験のデータなど,論文の本文中に載せられなかったが,
読者にとって役に立つとおもわれるデータなどを付録として掲載する.

\chapter{インタプリタの変換過程}

\begin{figure}
  % code/3cps/eval.ml より
\begin{verbatim}
(* CPS インタプリタを非関数化して CPS 変換した関数 *)
let rec eval (exp : e) (k : k) (k2 : k2) : a = match exp with
  | Val (v) -> apply_in k v k2
  | App (e1, e2) -> eval e2 (FApp2 (e1, k)) k2
  | Op (name, e) -> eval e (FOp (name, k)) k2
  | With (h, e) ->
    eval e FId (fun a -> apply_handler k h a k2)  (* GHandle に変換される *)

(* handle 節内の継続を適用する関数 *)
and apply_in (k : k) (v : v) (k2 : k2) : a = match k with
  | FId -> k2 (Return v)  (* 継続を適用 *)
  | FApp2 (e1, k) -> let v2 = v in eval e1 (FApp1 (v2, k)) k2
  | FApp1 (v2, k) -> let v1 = v in
    (match v1 with
      | Fun (x, e) ->
        let reduct = subst e [(x, v2)] in
        eval reduct k k2
      | Cont (cont_value) ->
        (cont_value k) v2 k2
      | _ -> failwith "type error")
  | FOp (name, k) ->
    k2 (OpCall (name, v, (fun v -> fun k2' -> apply_in k v k2'))) (* 継続を適用 *)

(* handle 節内の実行結果をハンドラで処理する関数 *)
and apply_handler (k : k) (h : h) (a : a) (k2 : k2) : a = match a with
  | Return v ->
    (match h with {return = (x, e)} ->
      let reduct = subst e [(x, v)] in
      eval reduct k k2)
  | OpCall (name, v, va) ->
    (match search_op name h with
      | None ->
        k2 (OpCall (name, v, (fun v -> fun k2' ->  (* 継続を適用 *)
          va v (fun a' -> apply_handler k h a' k2'))))  (* GHandle に変換 *)
      | Some (x, y, e) ->
        let cont_value =
          Cont (fun k'' -> fun v -> fun k2 ->
            va v (fun a' -> apply_handler k'' h a' k2)) in  (* GHandle に変換 *)
        let reduct = subst e [(x, v); (y, cont_value)] in
        eval reduct k k2)

(* 初期継続を渡して実行を始める *)
let interpreter (e : e) : a = eval e FId (fun a -> a)  (* GId に変換される *)
\end{verbatim}
\caption{CPS インタプリタを非関数化して CPS 変換したプログラム}
\label{figure:3cps}
\end{figure}

\begin{figure}
  % code/4defun/syntax.ml より
\begin{verbatim}
(* CPS インタプリタを非関数化して CPS 変換して非関数化した関数 *)
let rec eval (exp : e) (k : k) (k2 : k2) : a = match exp with
  | Val (v) -> apply_in k v k2
  | App (e1, e2) -> eval e2 (FApp2 (e1, k)) k2
  | Op (name, e) -> eval e (FOp (name, k)) k2
  | With (h, e) -> eval e FId (GHandle (h, k, k2))

(* handle 節内の継続を適用する関数 *)
and apply_in (k : k) (v : v) (k2 : k2) : a = match k with
  | FId -> apply_out k2 (Return v)
  | FApp2 (e1, k) -> let v2 = v in eval e1 (FApp1 (v2, k)) k2
  | FApp1 (v2, k) -> let v1 = v in (match v1 with
    | Fun (x, e) ->
      let reduct = subst e [(x, v2)] in
      eval reduct k k2
    | Cont (cont_value) ->
      (cont_value k) v2 k2
    | _ -> failwith "type error")
  | FOp (name, k) ->
    apply_out k2 (OpCall (name, v, (fun v -> fun k2' -> apply_in k v k2')))

(* 全体の継続を適用する関数 *)
and apply_out (k2 : k2) (a : a) : a = match k2 with
  | GId -> a
  | GHandle (h, k, k2) -> apply_handler k h a k2

(* handle 節内の実行結果をハンドラで処理する関数 *)
and apply_handler (k : k) (h : h) (a : a) (k2 : k2) : a = match a with
  | Return v -> (match h with {return = (x, e)} ->
    let reduct = subst e [(x, v)] in eval reduct k k2)
  | OpCall (name, v, va) ->
    (match search_op name h with
      | None ->
        apply_out k2 (OpCall (name, v,
          (fun v -> fun k2' -> va v (GHandle (h, k, k2')))))
      | Some (x, y, e) ->
        let cont_value =
          Cont (fun k'' -> fun v -> fun k2 -> va v (GHandle (h, k'', k2))) in
        let reduct = subst e [(x, v); (y, cont_value)] in
        eval reduct k k2)

(* 初期継続を渡して実行を始める *)
let interpreter (e : e) : a = eval e FId GId
\end{verbatim}
\caption{CPS インタプリタを非関数化して CPS 変換して非関数化したプログラム}
\label{figure:4defun}
\end{figure}

\chapter{学生の実行ログから得られたデータ}
\begin{table}[!b]
\begin{center}
  \begin{tabular}{|c||c|c|c|c|c|c||c|c|c|c|c|c||l|}
    \hline
    & \multicolumn{6}{|c||}{2017} & \multicolumn{6}{|c||}{2018} & \\ \cline{2-13}
    \hspace{-1mm}week\hspace{-1mm} & all & step. & try & \hspace{-1mm}mod.\hspace{-1mm} & \hspace{-1mm}print\hspace{-1mm} & ref
    & all & step. & try & \hspace{-1mm}mod.\hspace{-1mm} & \hspace{-1mm}print\hspace{-1mm} & ref & contents\\ \hline
    1 & 1293 & 504 & 0 & 0 & 0 & 0 & 1233 & 627 & 0 & 0 & 0 & 0 & fun.\ def.\\ \hline
    2 & 1511 & 235 & 0 & 0 & 0 & 0 & 1375 & 189 & 0 & 0 & 0 & 0 & if\\ \hline
    3 & 1618 & 144 & 0 & 0 & 0 & 0 & 1641 & 179 & 0 & 0 & 0 & 0 & record\\ \hline
    4 & 2364 & 169 & 0 & 0 & 0 & 0 & 2517 & 332 & 0 & 0 & 0 & 0 & list\\ \hline
    5 & 2556 & 193 & 0 & 0 & 0 & 0 & 3173 & 213 & 0 & 0 & 0 & 0 & list 2\\ \hline
    6 & 1596 & 43 & 0 & 0 & 0 & 0 & 1369 & 41 & 0 & 0 & 0 & 0 & Dijkstra\\ \hline
    7 & 2621 & 92 & 0 & 0 & 0 & 0 & 3570 & 86 & 0 & 0 & 0 & 0 & map\\ \hline
    8 & 1874 & 81 & 0 & 0 & 0 & 0 & 2028 & 75 & 0 & 0 & 0 & 0 & filter\\ \hline
    9 & 2184 & 34 & 0 & 0 & 0 & 0 & 3300 & 98 & 0 & 0 & 0 & 0 & gen.\ rec.\\ \hline
    10 & 2254 & 48 & 0 & 0 & 0 & 0 & 3298 & 106 & 3 & 0 & 0 & 0 & tree\\ \hline
    11 & 1783 & 20 & 10 & 0 & 0 & 0 & 2790 & 37 & 22 & 0 & 0 & 0 & exception\\ \hline
    12 & 1785 & 12 & 8 & 4 & 0 & 0 & 3501 & 37 & 3 & 26 & 3 & 0 & module\\ \hline
    13 & 1678 & 10 & 0 & 7 & 2 & 0 & 2943 & 22 & 0 & 3 & 16 & 0 & seq.\ exec.\\ \hline
    14 & 1280 & 11 & 0 & 0 & 0 & 1 & 1511 & 65 & 0 & 4 & 0 & 56 & ref\\ \hline
    15 & 517 & 6 & 0 & 0 & 0 & 0 & 1717 & 68 & 0 & 5 & 0 & 30 & heap\\ \hline
  \end{tabular}
\end{center}
  \caption{Number of uses of the stepper (step.) among all the
  executions (all) in 2017 and 2018.  The columns try, mod., print,
  and ref represent number of uses of the stepper for programs that
  contain exception handling, modules, printing (and sequential
  execution), and references (including arrays), respectively.
  The rightmost column shows representative topics handled in the
  week.}
  \label{TableUsage}
\end{table}

\begin{table}
\begin{center}  
\small
  \begin{tabular}{|l||c|c|c||c|c|c||l|}
    \hline
    week.
    & \multicolumn{3}{|c||}{2016 to 2017}
    & \multicolumn{3}{|c||}{2016 to 2018}
    &
    \\ \cline{2-7}
    problem & t & p & +/- & t & p & +/- & contents
    \\ \hline
    2.r1
    & t(55)=2.098 & p=\textcolor{red}{0.020} & dec
    & t(60)=0.635 & p=0.264 & dec
    & if\\ \cline{1-7}
    2.r2
    & t(56)=2.364 & p=\textcolor{red}{0.011} & dec
    & t(57)=1.831 & p=\textcolor{red}{0.036} & dec
    & \\ \cline{1-7}
    2.r3
    & t(54)=1.896 & p=\textcolor{red}{0.032} & dec
    & t(59)=0.751 & p=0.228 & dec
    & \\ \cline{1-7}
    2.1
    & t(66)=3.006 & p=\textcolor{red}{0.002} & dec
    & t(74)=3.372 & p=\textcolor{red}{0.001} & dec
    & \\ \cline{1-7}
    2.2
    & t(56)=3.672 & p=\textcolor{red}{0.000} & dec
    & t(62)=3.036 & p=\textcolor{red}{0.002} & dec
    & \\ \hline
    3.r1
    & t(52)=3.222 & p=\textcolor{red}{0.001} & dec
    & t(61)=2.936 & p=\textcolor{red}{0.002} & dec
    & record\\ \cline{1-7}
    3.r2
    & t(42)=2.339 & p=\textcolor{red}{0.012} & dec
    & t(56)=3.467 & p=\textcolor{red}{0.001} & dec
    & \\ \cline{1-7}
    3.r3
    & t(41)=1.373 & p=0.089 & dec
    & t(51)=2.688 & p=\textcolor{red}{0.005} & dec
    & \\ \cline{1-7}
    3.1
    & t(28)=5.610 & p=\textcolor{red}{0.000} & dec
    & t(38)=2.753 & p=\textcolor{red}{0.004} & dec
    & \\ \cline{1-7}
    3.2
    & t(17)=1.655 & p=0.058 & dec
    & t(27)=0.105 & p=0.459 & dec
    & \\ \cline{1-7}
    3.3
    & t(16)=1.546 & p=0.071 & dec
    & t(13)=0.603 & p=0.279 & dec
    & \\ \hline
    4.r1
    & t(47)=2.088 & p=\textcolor{red}{0.021} & dec
    & t(61)=2.446 & p=\textcolor{red}{0.009} & dec
    & list\\ \cline{1-7}
    4.r2
    & t(48)=1.909 & p=\textcolor{red}{0.031} & dec
    & t(60)=2.267 & p=\textcolor{red}{0.014} & dec
    & \\ \cline{1-7}
    4.1
    & t(51)=2.134 & p=\textcolor{red}{0.019} & dec
    & t(60)=2.473 & p=\textcolor{red}{0.008} & dec
    & \\ \cline{1-7}
    4.2
    & t(18)=3.033 & p=\textcolor{red}{0.004} & dec
    & t(20)=0.489 & p=0.315 & dec
    & \\ \hline
    5.r1
    & t(42)=1.037 & p=0.153 & dec
    & t(55)=0.257 & p=0.399 & inc
    & list 2\\ \cline{1-7}
    5.1
    & t(49)=1.592 & p=0.059 & dec
    & t(61)=0.904 & p=0.185 & dec
    & \\ \cline{1-7}
    5.2
    & t(55)=4.138 & p=\textcolor{red}{0.000} & dec
    & t(62)=1.631 & p=0.054 & dec
    & \\ \cline{1-7}
    5.3
    & t(47)=3.305 & p=\textcolor{red}{0.001} & dec
    & t(50)=1.940 & p=\textcolor{red}{0.029} & dec
    & \\ \hline
    6.r1
    & t(30)=0.322 & p=0.375 & inc
    & t(51)=2.011 & p=\textcolor{blue}{0.025} & inc
    & Dijkstra's algorithm\\ \cline{1-7}
    6.1
    & t(41)=1.678 & p=0.050 & dec
    & t(61)=1.155 & p=0.126 & dec
    & \\ \cline{1-7}
    6.2
    & t(45)=1.415 & p=0.082 & dec
    & t(62)=0.976 & p=0.166 & dec
    & \\ \cline{1-7}
    6.3
    & t(34)=2.296 & p=\textcolor{red}{0.014} & dec
    & t(42)=0.548 & p=0.293 & dec
    & \\ \hline
    7.r1
    & t(42)=0.462 & p=0.323 & inc
    & t(56)=0.314 & p=0.377 & dec
    & map\\ \cline{1-7}
    7.r2
    & t(41)=0.286 & p=0.388 & inc
    & t(54)=1.181 & p=0.121 & dec
    & \\ \cline{1-7}
    7.r3
    & t(40)=0.677 & p=0.251 & inc
    & t(51)=1.492 & p=0.071 & dec
    & \\ \cline{1-7}
    7.1
    & t(21)=0.965 & p=0.173 & dec
    & t(20)=0.372 & p=0.357 & dec
    & \\ \cline{1-7}
    7.2
    & t(12)=0.380 & p=0.355 & inc
    & \, t(7)=0.686 & p=0.258 & dec
    & \\ \hline
    8.r1
    & t(46)=1.162 & p=0.126 & inc
    & t(58)=2.694 & p=\textcolor{red}{0.005} & dec
    & filter\\ \cline{1-7}
    8.1
    & t(16)=0.844 & p=0.205 & dec
    & t(22)=0.841 & p=0.205 & dec
    & \\ \hline
    9.r1
    & t(44)=1.294 & p=0.101 & dec
    & t(55)=0.678 & p=0.250 & dec
    & general recursion\\ \hline
    10.r1
    & t(48)=0.312 & p=0.378 & dec
    & t(51)=1.308 & p=0.098 & dec
    & tree\\ \cline{1-7}
    10.r2
    & t(48)=0.457 & p=0.325 & dec
    & t(50)=1.760 & p=\textcolor{red}{0.042} & dec
    & \\ \cline{1-7}
    10.1
    & t(33)=0.976 & p=0.168 & dec
    & t(38)=0.845 & p=0.202 & dec
    & \\ \cline{1-7}
    10.2
    & t(19)=1.498 & p=0.075 & dec
    & t(19)=0.871 & p=0.197 & dec
    & \\ \cline{1-7}
    10.3
    & t(15)=1.538 & p=0.072 & dec
    & t(12)=2.240 & p=\textcolor{red}{0.022} & dec
    & \\ \hline
    11.r1
    & t(43)=0.272 & p=0.393 & dec
    & t(53)=1.555 & p=0.063 & dec
    & exception\\ \cline{1-7}
    11.r2
    & t(37)=0.454 & p=0.326 & dec
    & t(44)=1.906 & p=\textcolor{red}{0.032} & dec
    & \\ \cline{1-7}
    11.r3
    & t(31)=0.050 & p=0.480 & inc
    & t(40)=1.208 & p=0.117 & dec
    & \\ \cline{1-7}
    11.1
    & t(34)=1.567 & p=0.063 & dec
    & t(46)=1.518 & p=0.068 & dec
    & \\ \cline{1-7}
    11.2
    & t(28)=2.204 & p=\textcolor{red}{0.018} & dec
    & t(36)=1.563 & p=0.063 & dec
    & \\ \cline{1-7}
    11.3
    & t(17)=0.604 & p=0.277 & dec
    & t(21)=0.384 & p=0.352 & dec
    & \\ \hline
    12.r1
    & t(39)=2.229 & p=\textcolor{red}{0.016} & dec
    & t(52)=4.009 & p=\textcolor{red}{0.000} & dec
    & module\\ \hline
    all
    & t(1778)=2.819 & p=\textcolor{red}{0.002} & dec
    & t(2111)=2.592 & p=\textcolor{red}{0.005} & dec
    & \\ \hline
  \end{tabular}
  \end{center}
  \caption{Result of one-sided t-test with p-values comparing the time between the beginning of the class and the moment that students submitted a correct solution.  The column +/- shows whether the average time increased or decreased.  The p-values below $0.05$ are colored.}
  \label{TableTTest}
\end{table}
