\chapter{研究内容}
自分がおこなった研究について説明をおこなう.この部分は必要ならば,章を増やして
より詳細に説明をおこなう.この章のタイトルとなっている「研究内容」は自分
の研究内容をよく表すタイトルに変更すること.

\section{table環境}
figure環境と同じように表の位置および表の番号,タイトルをつけるときには
table環境を使います.

表を配置するおおよその位置は,\verb/\begin{table}/の後に指定します.
\verb/\begin{table}[h]/では,その位置に入ります.(オプションは\verb+figure+環境と同じです.)
\verb/\caption/は,表番号と表タイトルを付けます.
\verb/\label/は,表番号を参照するためのラベル(名前)です.
表番号は自動的に付きます.図を挿入する場合とのひとつだけの違いは,\verb+\caption+は図の場合は,図の下に入れますが,表の場合は表の上に入れます.このことは一般的に文書を作成するときの慣例となっているので注意してください.\\

表\ref{table_sample}は,表の挿入例です.

\begin{table}[h]
\begin{center}
\caption{表の挿入例}
\begin{tabular}{|l|l|}
\hline
役割 & 氏名\\
\hline
\hline
内閣総理大臣 & 小泉純一郎\\
\hline
総務大臣 & 麻生太郎\\
\hline
法務大臣 & 野沢太三\\
\hline
外務大臣 & 川口順子\\
\hline
財務大臣 & 谷垣禎一\\
\hline
文部科学大臣 & 河村建夫\\
\hline
\end{tabular}
\end{center}
\label{table_sample}
\end{table}

{\bf 注意!表や図を挿入し,参照機能を使った場合は必ずコンパイルを2回は
してください.参照機能は2回コンパイルをしないときちんと働いてくれません!}