\chapter{5}
\label{chapter:incremental}

\input{5/spec}

\input{5/attribute}

\input{5/solution}

\section{$\lambda$計算に対する実装}

\begin{figure}
\begin{verbatim}
(* 式の種類の定義 *)
type expression_desc = Var of string                   (* x *)
                     | Fun of string * expression      (* λx. e *)
                     | App of expression * expression  (* e1 e2 *)
                              
and expression = {desc : expression_desc;  (* 式の内容 *)
                  attr : attribute list}   (* attribute [@name ... ] *)

and attribute = (string * payload)       (* 名前と内容のペア *)
and payload = (int * expression) option  (* ステップ番号と簡約前の式 *)
\end{verbatim}
\caption{対象言語の定義}
\label{figure:lambda}
\end{figure}

本節では、既存の OCaml ステッパ\cite{FSA18}の実装を紹介し、新しい OCaml ステッパの実装を示す。対象言語は型無しの$\lambda$式で、さらに実際の OCaml を模して任意の部分式に複数の attribute を付けられるものとするが、各 attribute の第一引数は attribute の名前とし、第二引数には\ref{生じる問題と解決方法-情報の消失-解決方法}節で定めた簡約の情報を簡単に表すために \texttt{Some (整数, 式)} または情報を持たないことを示す \texttt{None} のどちらか(\texttt{payload} 型)をとる\footnote{\texttt{[@x ]} は \texttt{("x", None)}、\texttt{[@stepper.reduct (1, 5 * 7)]} は \texttt{("stepper.reduct", Some (1, 5 * 7))} である。}。これらの型を図\ref{figure:lambda}のように定義する。式は \texttt{expression} 型であり、式本体の内容を表す \texttt{desc} と任意の個数の attribute を表す \texttt{attr} の2つの要素を持つ。

\subsection{incremental でないステッパ関数}

\begin{figure}[t]
\begin{spacing}{0.8}
  \begin{alltt}
\colorbox{lightgray}{(* コンテキストのフレームの定義 *)}
\colorbox{lightgray}{type frame = AppR of expr  (* e [.] *)}
\colorbox{lightgray}{           | AppL of expr  (* [.] v *)}

\colorbox{lightgray}{(* ステップ番号を格納する変数 *)}
\colorbox{lightgray}{let counter : int ref = ref 0}

(* 式\colorbox{lightgray}{とその周りのコンテキスト}を受け取って式を評価する *)
let rec eval (expr : expression) \colorbox{lightgray}{(context : frame list)} : expression =
  match expr.desc with
    | Var (x) -> failwith "error: Unbound variable"
    | Fun (x, f) -> expr
    | App (e1, e2) ->
      let arg\_value = eval e2 \colorbox{lightgray}{(AppR e1 :: context)} in       (* 引数部分を評価 *)
      let fun\_value = eval e1 \colorbox{lightgray}{(AppL arg\_value :: context)} in(* 関数部分を評価 *)
      match fun\_value.desc with
        | Fun (x, f) ->
          let redex = \{desc = App (fun\_value, arg\_value);       (* 簡約前の式 *)
                      attr = expr.attr\} in
          let reduct = \{desc = (subst f x arg\_value).desc ;     (* 簡約後の式 *)
                        attr = expr.attr\} in
          \colorbox{lightgray}{memo redex reduct context;                          (* ステップ出力 *)}
          eval reduct \colorbox{lightgray}{context}                             (* 簡約後の式を評価 *)
        | \_ -> failwith "error: not a function"

(* \colorbox{lightgray}{空のコンテキストで}式の評価を始める *)
let start (expr : expression) : expression = eval expr \colorbox{lightgray}{[]}
\end{alltt}
\end{spacing}
\caption{incremental でないステッパ関数の実装}
\label{figure:old-stepper}
\end{figure}

incremental でないステッパ関数は big-step インタプリタ関数に
ステップ出力のための作用を追加することで構築されている。
incremental でないステッパ関数の実装は図\ref{figure:old-stepper}のようになる。
関数 \texttt{eval} は OCaml の call-by-value かつ right-to-left
の評価戦略に従った代入ベースの $\lambda$
計算のインタプリタにステップ実行のための作用を追加したステッパ関数である。
背景に灰色が付いた部分がステップ実行のための作用であり、
白い部分のみを読むと単なるインタプリタとして見ることができる。
ただし関数 \texttt{subst} は代入の関数であり、
\texttt{subst f x arg\_value} は式 \texttt{f} の中の変数
\texttt{x} を式 \texttt{arg\_value} に置換した式を返す。
このステッパ関数の出力は、
例えば入力プログラム \texttt{2 * 3 + 5 * 7} に対して、
図 \ref{figure:highlight} の左側の文字列(にステップ番号の表示を足したもの)である。

ステッパ関数は実行可能なプログラムのみを受け付けるため、
ステッパ関数に渡される式の中に自由変数および型エラーは存在しない。
さらにこのステッパ関数の基となる代入ベースのインタプリタでは、
関数の内部の式は必ず関数適用の簡約(すなわち実引数の代入)の後に実行するので、
常に変数はその実行の前に束縛を解決されており、変数がインタプリタ関数の引数として実行されることはない。
よって、関数 \texttt{eval} 中の \texttt{failwith} の呼び出しは起こり得ない。

\begin{figure}[t]
\begin{spacing}{0.8}
\begin{alltt}
(* 簡約前後の式とコンテキストを受け取って、そのステップを出力する *)
let memo (redex : expression) (reduct : expression) (context : frame list)
  : unit =
  let marked\_redex =                            (* 簡約前の式に attribute 追加 *)
    \{redex with attr = Some ("stepper.redex", None)\} in
  let marked\_reduct =                           (* 簡約後の式に attribute 追加 *)
    \{reduct with attr = Some ("stepper.reduct", None)\} in
  print\_counter ();                                 (* 簡約前ステップ番号を出力 *)
  print (plug redex current\_context);               (* 簡約前のプログラムを出力 *)
  counter := !counter + 1;                          (* ステップ番号を 1 増やす *)
  print\_counter ();                                 (* 簡約後ステップ番号を出力 *)
  print (plug reduct current\_context)               (* 簡約後のプログラムを出力 *)
\end{alltt}
\end{spacing}
\caption{ステップ出力関数}
\label{figure:memo}
\end{figure}

関数 \texttt{eval} の下から3行目の関数 \texttt{memo} は、簡約前のプログラムを出力し、 \texttt{counter} の値を1増やし、簡約後のプログラムを出力する関数である。その実装は図\ref{figure:memo}に示す。ただし、関数 \texttt{print\_counter : unit -> unit} はコメントとしてステップ番号 \texttt{(* Step n *)} を標準出力する関数、関数 \texttt{print : expr -> unit} は式を標準出力する関数、関数 \texttt{plug : expression -> frame list -> expression} は計算している途中の部分式とコンテキストを受け取って式を再構成する関数であり、実装は省略する。変数 \texttt{counter} には、現在のステップ番号が格納されている。式の評価中に関数適用の簡約を行うたびに1ずつ増加させることで、式全体の通しステップ番号を出力できる。


\input{5/new_implementation}

\section{実際の OCaml ステッパ}
\label{section:ocaml stepper}

我々が実装した OCaml ステッパは、
\ref{section:try-with__stepper} 節で示した try-with と型無し $\lambda$ 計算に対するステッパと比べて
以下のような特徴がある。

\begin{itemize}
\item 型エラー等を想定しない。
\item 授業「関数型言語」で使用する構文のほぼ全てに対応している。
\item 関数適用が $\beta$ 簡約されるステップからその式が値になるステップまで飛ばす機能がある。
\end{itemize}

この節ではそれぞれについて述べる。

\subsection{型エラーについて}
\label{subsection:stepper__type}

OCaml ステッパでは、
まず入力プログラムを OCaml のパーザを利用して構文木にし、型チェックをする。
未定義変数エラーを含むシンタックスエラーや型エラーになった場合は
ステッパ本体のプログラムを起動せず、OCaml が示すエラーメッセージを表示する。
よって、コンパイルエラーがあるプログラムの構文木がステッパ関数に渡されることは無いという
仮定の上で実装されている。
ゼロ除算等の実行時エラーに関しては OCaml では例外の発生として処理されるので、
そのようにステップ実行処理を続行する。
ただし、例外が捕捉されないままその文の実行が終わってしまった場合はそこでステップ実行を終了する。

\subsection{対象構文}
\label{subsection:stepper__syntax}
% In Figure \ref{figure:ocamlstep},
% we show a reduction sequence produced by the actual stepping evaluator.
% 我々が実装した OCaml ステッパが出力するステップ列の例を図 \ref{figure:ocamlstep} に示す。
% The evaluator supports the following syntactic constructs:
このステッパは以下の構文に対応している。

\begin{itemize}
% \item integers, floating point numbers, booleans, characters, strings
% \item lists, tuples, records
% \item user-defined datatypes
% \item conditionals, let-expressions, recursive functions, pattern-matching
% \item exception handling operators
% \item printing functions and sequential execution
% \item the List module, user-defined modules
% \item references, arrays
\item 整数、実数、真偽値、文字、文字列型
\item リスト、組、レコード
\item ユーザ定義型
\item 条件分岐、変数定義、再帰関数定義、パターンマッチ
\item List モジュール、ユーザ定義モジュール
\item 例外処理
\item 標準出力関数、逐次実行
\item 書き換え可能な変数、配列
\end{itemize}

標準出力や書き換え可能な変数を含むプログラムでは、
標準出力された文字列や変数に格納された値などの「状態」をインタプリタが保持する必要がある。
状態の情報はステッパプログラム内の書き換え可能なグローバル変数の中に格納することで実装した。

\begin{figure}
  \includegraphics[width=14cm]{6/longexample.eps}
%   \caption{Evaluating programs using the actual stepper}
  \caption{実際のステッパでプログラムを実行する様子}
  \label{figure:ocamlstep}
\end{figure}

OCaml ステッパは他に授業で利用する一部の演算子などに対応している。
図 \ref{figure:ocamlstep} に実際のステップ列の例を示す。

% ref、配列を使うやつにする?

\subsection{関数適用のスキップ}
\label{subsection:skip}

\begin{figure}
  \includegraphics[width=10cm]{6/beforeskip.eps}
  \includegraphics[width=4.3cm]{6/afterskip.eps}
%   \caption{Skipping evaluation of the factorial function}
  \caption{階乗関数をスキップする様子}
  \label{figure:factskip}
\end{figure}

% To allow the user to adjust granularity of steps,
% we provide an option for skipping
% the evaluation of the current function application.
プログラムのステップ数が多くなると、デバッグの目的でステップ実行をするときに
膨大な計算ステップを1つ1つ追って見るのは効率が悪く、
次第に実用に耐えるものではなくなってしまう。
そこで本研究では、「関数適用」を基準としてステップを飛ばす機能を追加した。

% Let us look at Figure \ref{figure:factskip},
% which shows skipping of the factorial function.
図 \ref{figure:factskip} に、階乗関数のスキップの様子を示す。
% By pressing the ``skip'' button,
% we can directly go from the program on the left to the one on the right,
% without seeing the intermediate steps that appear
% during the evaluation of the function's body.
左の状態で「skip」ボタンを押すと、関数の中身
\texttt{if 3 = 0 then 1 else 3 * (factorial (3 - 1))} が
\texttt{6} になるまでの実行ステップを見ることなく右の画面に移ることができる。
% This feature helps us focus on the steps we are interested in,
% allowing us to grasp the overall flow of the execution.
これによって見たいステップだけに注目することができ、実行の全体の流れを把握しやすくなる。

% The skipping feature requires some modifications to
% the \texttt{eval} function (Figure \ref{figure:skipapp}).
このスキップ機能は、
\ref{chapter:try-with} 章で try-with に対応したステッパ関数
(図 \ref{figure:stepper}) の eval 関数を
図 \ref{figure:skipapp} のように変更することで実装できる。
% The idea is to sandwich the steps within an application between two strings:
本研究では、飛ばすステップ列を
% \texttt{(* Application n start *)} and \texttt{(* Application n end *)}.
\texttt{(* Application n start *)} と \texttt{(* Application n end *)}
という 2 つの文字列で挟む方法をとった。
% Here, \texttt{n} tells us at which step we have entered the application.
\texttt{n} は関数適用の実行が始まるステップの番号である。
あるステップで実行が始まる関数適用は必ず 1 つ以下なので、
このステップ番号によって関数適用を一意に特定することができる。
% These strings are printed using the \texttt{apply\US start} and
% \texttt{apply\US end} functions,
% and help the Emacs Lisp program to hide unnecessary steps.
関数 \texttt{apply\US start :\ unit -> int} が前者を出力する関数、
関数 \texttt{apply\US end :\ int -> unit} が後者を出力する関数である。
表示処理をする Emacs Lisp プログラムがこれらの文字列を検索し、
間のステップを隠す処理をする。
% We show an example output sequence in Figure \ref{figure:skipping}.
スキップ機能を追加したステッパ関数の出力は例えば図 \ref{figure:skipping} のようになる。

\begin{figure}
\begin{alltt}
let rec eval expr ctxt = match expr with
    ...
  | App (e1, e2) ->
    begin
      let v2 = eval e2 (add ctxt (CAppR e1)) in
      let v1 = eval e1 (add ctxt (CAppL v2)) in
      match v1 with
        | Lam (x, e) ->
          let e' = subst e x v2 in
          \colorbox{lightgray}{let apply_num = apply_start () in}               (* 開始マークを出力 *)
          memo (App (v1, v2)) e' ctxt;
          let v = eval e' ctxt in
          \colorbox{lightgray}{apply_end apply_num;}                            (* 終了マークを出力 *)
          v
        | _ -> failwith "not a function"
    end
  | ...
\end{alltt}
\caption{関数適用をスキップするためのステッパ関数}
\label{figure:skipapp}
\end{figure}

\begin{figure}
\texttt{(* Step 0 *) (f 4) + \colorbox{lightgreen}{10 * 100}\\
(* Step 1 *) (f 4) + \colorbox{purple}{1000}\\
(* Application 1 start *)\\
(* Step 1 *) \colorbox{lightgreen}{f 4} + 1000\\
(* Step 2 *) \colorbox{purple}{(4 * 2) - 1} + 1000\\
(* Step 2 *) \colorbox{lightgreen}{(4 * 2} - 1) + 1000\\
(* Step 3 *) \colorbox{purple}{(8} - 1) + 1000\\
(* Step 3 *) \colorbox{lightgreen}{8 - 1} + 1000\\
(* Step 4 *) \colorbox{purple}{7} + 1000\\
(* Application 1 end *)\\
(* Step 4 *) \colorbox{lightgreen}{7 + 1000}\\
(* Step 5 *) \colorbox{purple}{1007}}
\caption{関数適用をスキップするためのステッパ関数の出力}
\label{figure:skipping}
\end{figure}


\subsection{ツールの実装}

ここまで、「プログラムを入力されてステップ実行の文字列を出力するプログラム」としてのステッパプログラムの実装を紹介した。ユーザが incremental なステッパツールを使用するには、ユーザの入力を受けてステッパプログラムを呼び出しステップを表示する外部のプログラムが必要になる。

DrRacket のステッパ\cite{clements01}や incremental でない OCaml ステッパでは、
外部のプログラムは「ステッパを起動して、出力を蓄えて、ユーザの操作に従って表示」をしていたが、
本研究のステッパでは、
「ユーザの操作に従ってステッパを呼び出して、出力されたものを装飾して表示」をする。
すると、外部のプログラムではステップ番号と前回の出力のみを保持することで実装が可能になる。


\section{予想される問題点とその経過}
本節では、incremental な OCaml ステッパを実際に利用するときに発生しうる問題点とその解決方法を挙げ、
大学の授業 ( \ref{section:experiment__course} 節で触れる) で利用してそれらが実際に問題になったかどうかを述べる。

\subsection{文字数の爆発}

\subsubsection{問題点}

本章のステッパでは、
任意のステップの「処理用の出力」に入力プログラムからそこまでの簡約の過程が記されているため、
1ステップ進むごとに文字数が増加する。
具体的には、簡約基 \texttt{e1} が式 \texttt{e2} に簡約されるステップでは、
その時点のコンテキストを \texttt{E}、ステップ番号を \texttt{n} とすると、
簡約前のプログラムが \texttt{E[e1]}、
簡約後のプログラムが \texttt{E[e2[@stepper.reduct n;; e1]]} となる。
\texttt{E} の文字数は変わらないので、\texttt{e2[@stepper.reduct n;;}
と \texttt{]}の分の文字数が増加することになる。
これを続けていくと、1ステップあたり少なくとも22文字は増加することになり、
仮に百万ステップの簡約をするとプログラムは数千万文字になる。
すると、毎ステップの入出力や通信に時間がかかる可能性がある。

\subsubsection{解決方法}

解決策としては、古いステップの attribute は削除してしまうという方法が考えられる。
たくさんのステップを見てから最初の方のステップまで戻るユーザは少ないと仮定すれば、
ある程度前のステップについての attribute があったら、
関数 \texttt{memo} で出力するプログラムを再構成する際に消去すれば、
さほど実際の使用に影響なく出力する文字数を減らすことができる。

\subsubsection{使用した結果}

実際の incremental な OCaml ステッパ (\ref{実装-実際のステッパ} 節) では、
Emacs Lisp プログラムによってステッパ関数の実行や表示を制御する。
これを用いた授業では、文字数の多さが原因の不具合は報告されず、
1 ステップの入出力にかかる時間も利用に支障が出ない程度だった。

\subsection{実行時間}
\label{予想される問題点-実行時間}

\subsubsection{問題点}

ステップ数が膨大になると、後ろの方までステップ実行をするのは困難である。
incremental なステッパ関数での実行速度は通常の OCaml 処理系に決して及ばないので、
1ステップずつ進める場合でも、スキップ機能(\ref{実装-実際のステッパ}節)を使う場合でも、
実行を多く進めるには長い時間が掛かってしまう。

\subsubsection{解決方法}
少しでも急ぐ為には、一般的なデバッガのように、
ステップ実行したい式の付近にユーザがブレークポイントを設定して、
そこからステップ実行を始めるという方法が考えられる。
そのためには、ブレークポイントまでを部分的に native code にコンパイルして実行するなどの方法を取らざるを得ない。
しかしいずれにしても、
通常の OCaml コンパイラを用いても長時間かかるプログラムの実行を早く終わらせることは不可能である。

\subsubsection{使用した結果}

1 ステップずつ進める場合は、プログラムの実行を後ろの方まで進めることは困難であるが、
ステップごとの実行時間が問題になることはなかった。
スキップ機能を使用する場合は、やはり通常の OCaml で実行するよりも長い時間を要した。
具体的には、東京メトロ全線の駅についてダイクストラ法で最短路問題を解くプログラムの実行に、
通常の OCaml では 0.327 秒、incremental な OCaml ステッパではスキップ機能を使用して約 54 秒かかった。

\subsection{関数適用評価スキップ後の前ステップ出力}

\subsubsection{問題点}

関数適用をスキップ(\ref{実装-実際のステッパ}節)した後に前のステップに戻ろうとすると、
スキップで飛ばされたステップには戻ることができない。
たとえば図\ref{figure:skip}のように2の階乗の計算をスキップしたとすると、
図\ref{figure:skip}の右の状態から、incremental でないステッパでは
\begin{itemize}
\item スキップをする前の関数適用式が簡約されるステップ5(図\ref{figure:skip}左)
\item 関数適用式が最終的な値になるステップ17(
\texttt{3 * \colorbox{lightgreen}{(2 * 1)} $\leadsto$ 3 * \colorbox{purple}{2}})
\end{itemize}
のどちらにも戻ることができたが、incremental なステッパでは前者にしか戻ることができない。

その原因は、incremental でないステッパでは \ref{予想される問題点-実行時間}
節で述べたように文字列検索によってステップ表示を切り替えていたので任意のステップに移ることができたのに対して、
incremental なステッパではステッパ関数が関数適用式が値になるまでを1ステップとして出力し、
その間の簡約についての情報は出力しないからである。

\subsubsection{解決方法}

これを解決するには、
スキップする部分の計算をしている間の簡約についても attribute に情報を蓄え、
スキップ後のステップで全ての簡約の情報が入った長いプログラムを出力する必要がある。
しかしそのようにすると
\ref{予想される問題点-文字数の爆発} 節と
\ref{予想される問題点-実行時間} 節の問題がより深刻になる可能性がある。

\subsubsection{使用した結果}

ステッパを利用した学生に答えてもらったアンケートに
「ステッパの不便なところ、欲しい機能などがあれば教えてください。」
という質問を含めたが、この点に関する要望は出なかった。


\section{まとめと今後の課題}

本研究では、OCaml のプログラムの簡約による書き換えを1ステップずつ見せるプログラムであるステッパを、1度の実行で1つの簡約のみを行うように変更し、ステッパの起動時間を短縮した。またそのためにあるステップのプログラムからそれ以前のステップのプログラムを計算できるようにする必要が生まれたので、それまでの簡約の内容を全て記録したプログラムを出力することで入力プログラムの情報が失われないように変更した。

今後は、一部の副作用を伴うプログラムに対応することを目指す。incremental でない OCaml ステッパ\cite{FSA18} はこれらに対応しているが、incremental になり、ストアや標準出力された文字列の情報をステッパのプロセスが保持できなくなったことで、ストアの情報もステッパの出力に含める必要が生じた。これは恐らく、プログラムの attribute (\ref{OCamlのattribute-プログラムのattribute}節)を利用することで解決できるであろう。
